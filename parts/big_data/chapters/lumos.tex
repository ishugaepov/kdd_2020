\chapter{Lumos: A Library for Diagnosing Metric Regressions in Web-Scale Applications}

Microsoft \\

\textbf{Reference:}~\url{https://arxiv.org/abs/2006.12793}

\textbf{Keywords:} online metrics, A/B experiments, anomaly detection

\textbf{Code:}~\url{https://github.com/microsoft/MS-Lumos}

\section*{Какую задачу решают авторы?}

Неотъемлемой частью любого web-scale приложения является система мониторинга, которая отслеживает большое количество метрик в течении времени, и сообщает об аномалиях.

Быстрое обнаружение аномалий и возможных причин их возникновения очень важно для стабильной работы сервиса и хорошего пользовательского опыта. \\

Существующие state-of-the-art алгоритмы детекции аномалий имеют довольно высокий false-positive rate и сами по себе не могут дать ответ на вопрос, что могло послужить причиной возникновения аномалии.

Уменьшение доли ложных срабатываний и помощь в поиске истинной причины появления аномалии могут существенно сэкономить время разработчиков. \\

В данной статье авторы представляют Python фрэймворк Lumos, который решает обозначенные проблемы.

% \section*{Как решают}

% \section*{Преимущества подхода}

% \section*{Мое мнение}