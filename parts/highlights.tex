Этот документ содержит мои заметки о некоторых статьях и туториалах с прошедшего KDD.

Заметки оформлены в формате конспектов по типу тех, которые можно найти тут \url{https://vk.com/papersreaders}. \\

В документе 11 полноценных конспектов и еще 10 мини конспектов. \\

Для себя я выделил несколько направлений, на работы в которых я смотрел в первую очередь:
\begin{itemize}
    \item Advertising --- все что связано с рекламой
    \item Recommender Systems --- задачи связанные с рекомендательными системами
    \item Deep Learning
\end{itemize}

Также на конференции было много статей и туториалов, посвещенных обучению на графах.

\section*{Highlights}

Так как конференция проходила в удаленном формате, то для большинства туториалов были предзаписаны видео, поэтому получилось посмотреть записи почти всех туториалов, которые меня заинтерисовали. \\

Ниже список туториалов, которые показались мне наиболее интересными, с кратким описанием.

\paragraph{Advances in Recommender Systems: From Multi-stakeholder Marketplaces to Automated RecSys} $ $\\

\url{https://sites.google.com/view/kdd20-marketplace-autorecsys/}

Some text about tutorial

\paragraph{Causal Inference Meets Machine Learning} $ $\\

\url{http://kdd2020tutorial.thumedialab.com}

Some text about tutorial

\paragraph{Neural Structured Learning} $ $\\

Some text about tutorial

\paragraph{Scalable Graph Neural Networks with Deep Graph Library} $ $\\

Some text about tutorial

\paragraph{Building Recommender Systems with PyTorch} $ $\\

\url{https://github.com/pytorch/workshops/tree/master/KDD_2020}

Some text about tutorial

\paragraph{Deep Learning for Search and Recommender Systems} $ $\\

\url{https://sites.google.com/view/kdd20tutorial-deepsnr}

Some text about tutorial

% \paragraph{Robust Deep Learning Methods for Anomaly Detection} $ $\\

% Some text about tutorial

