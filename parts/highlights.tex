Этот документ содержит мои заметки о некоторых статьях и туториалах с прошедшего KDD.

Заметки оформлены в формате конспектов по типу тех, которые можно найти тут \url{https://vk.com/papersreaders}. \\

В документе 11 полноценных конспектов и еще 10 мини конспектов. \\

Для себя я выделил несколько направлений, на работы в которых я смотрел в первую очередь:
\begin{itemize}
    \item Advertising --- все что связано с рекламой
    \item Recommender Systems --- задачи связанные с рекомендательными системами
    \item Deep Learning
\end{itemize}

Также на конференции было много статей и туториалов, посвещенных обучению на графах.

\chapter{Tutorials}

Так как конференция проходила в удаленном формате, то для большинства туториалов были предзаписаны видео, поэтому получилось посмотреть записи почти всех туториалов, которые меня заинтерисовали. \\

Ниже список туториалов, которые показались мне наиболее интересными, с кратким описанием.

% \section*{Advances in Recommender Systems: From Multi-stakeholder Marketplaces to Automated RecSys} 

% \url{https://sites.google.com/view/kdd20-marketplace-autorecsys/}

% Some text about tutorial

% \section*{Causal Inference Meets Machine Learning} 

% \url{http://kdd2020tutorial.thumedialab.com}

% Some text about tutorial

\section*{[Amazon AWS] Scalable Graph Neural Networks with Deep Graph Library} 

Link:~\url{https://github.com/dglai/KDD20-Hands-on-Tutorial}

Paper:~\url{https://arxiv.org/pdf/1909.01315.pdf} \\

Очень практический туториал про обучение графовых нейросетей. 

Речь в основном идет про обучение GraphSAGE\footnote{\url{https://vk.com/papersreaders?w=wall-154085965_143}} с помощью фрэймворка DGL\footnote{\url{https://www.dgl.ai}} на графах разного размера: начиная с маленьких графов и обучения на CPU, заканчивая обучением на огромных графах с использованием нескольких машин. \\

Туториал наглядно показывает, что обучать GNN используя DGL довольно просто.

\begin{enumerate}
    \item Краткий обзор задач и методов в GNNs --- \url{https://www.youtube.com/watch?v=Soa-66W1CEQ}
    \item Вводное видео --- \url{https://www.youtube.com/watch?v=JMF4KAkO0zo}
    \item Обучение GraphSAGE с помощью DGL+PyTorch на задачах node classification и link prediction (на gpu в том числе) --- \url{https://www.youtube.com/watch?v=OEBk_HvKNcQ}
    \item Обучение GraphSAGE на больших графах (сэмплирование соседей, dataloader'ы) на node classification и link prediction (single machine - single/multiple gpu(s)) --- \url{https://www.youtube.com/watch?v=2W0UIJX3GV8}
\end{enumerate}

\section*{Graph Representation Learning}

Video:~\url{https://www.youtube.com/watch?time_continue=27&v=J9AhyrTs2Nk&feature=emb_logo}

Book:~\url{https://www.cs.mcgill.ca/~wlh/grl_book/} \\

Обзорный рассказ об истории развития GRL, о современных методах и приложениях, о текущих челенджах и открытых вопросах.

% \section*{Building Recommender Systems with PyTorch} 

% \url{https://github.com/pytorch/workshops/tree/master/KDD_2020}

% Some text about tutorial

\section*{[Linkedin] Deep Learning for Search and Recommender Systems} 

Link:~\url{https://sites.google.com/view/kdd20tutorial-deepsnr} \\

Туториал от Linkedin про DL методы в поиске и рекомендациях: обзор основных задач, методы решений и Hands On с использованием их фрэймворка DeText\footnote{\url{https://github.com/linkedin/detext}}.

\begin{enumerate}
    \item Краткий обзор туториала --- \url{https://www.youtube.com/watch?time_continue=2&v=Dp2Iwd4FWsU&feature=emb_logo}
    \item Обзор задач возникающих в поиске, и методов их решения --- \url{https://www.youtube.com/watch?v=LwjIKqNsQjc&feature=emb_logo}. В основе всего Sequential Modeling. 
    Задачи
        \begin{itemize}
            \item Query Intent Classification (есть в DeText) - понять к чему относится запрос, поиск людей/вакансий/постов итд.
            \item Query Completion (есть в DeText) - автодополнение поискового запроса
            \item Query Suggestion - рекомендация запросов
            \item Sequence Labeling - выделение в тексте запроса сущностей (NER)
            \item Ranking (есть в DeText) - ранжирование отобранных документов-кандидатов
        \end{itemize}
    \item Hands On: query intent classification with DeText --- \url{https://www.youtube.com/watch?v=QZ5fbO3NA64&feature=emb_logo}
    \item Hands On: query completion with DeText --- \url{https://www.youtube.com/watch?v=OLfeOcZ7y6A&feature=emb_logo}
    \item Candidate Retrieval: обзор классических (lexical matching) и современных (embeddings based) методов отбора документов-кандидатов --- \url{https://www.youtube.com/watch?v=KECp0syOH28&feature=emb_logo}
    \item Hands On: candidates retrieval with Elasticsearch --- \url{https://www.youtube.com/watch?v=MXbttPyNON0&feature=emb_logo}
    \item Ranking: Metrics (NDCG, MAP, MRR), Learning to Rank (loss functions, algorithms) --- \url{https://www.youtube.com/watch?v=mw_zdRLbhJM&feature=emb_logo}
    \item Hands On: Learning to Rank with DeText --- \url{https://www.youtube.com/watch?v=C-erioRqg4Y&feature=emb_logo}
\end{enumerate}

\section*{[Google] Neural Structured Learning} 

Практический туториал про регуляризацию обучения с помощью информации представленной в виде графов (реализация Graph-RISE\footnote{\url{https://vk.com/papersreaders?w=wall-154085965_180}}) на TF. \\

References:

\begin{itemize}
    \item Neural Graph Learning: Training Neural Networks Using Graphs\footnote{\url{https://storage.googleapis.com/pub-tools-public-publication-data/pdf/bbd774a3c6f13f05bf754e09aa45e7aa6faa08a8.pdf}}
	\item Graph-RISE: Graph-Regularized Image Semantic Embedding\footnote{\url{https://arxiv.org/pdf/1902.10814.pdf}}
\end{itemize}

Videos:

\begin{enumerate}
    \item \url{https://www.youtube.com/watch?v=iyONhiDuCKE&feature=emb_logo}
    \item \url{https://www.youtube.com/watch?v=MQ3Ec-19_bc}
    \item \url{https://www.youtube.com/watch?v=qmY-CANCTdk&feature=emb_logo}
    \item \url{https://www.youtube.com/watch?v=h-nTn4nWZo8&feature=emb_logo}
    \item \url{https://www.youtube.com/watch?v=qTC470NNlZ0&feature=emb_logo}
    \item \url{https://www.youtube.com/watch?v=9Mwf0VaWGwE&feature=emb_logo}
    \item \url{https://www.youtube.com/watch?v=FqOWxvvYSA8&feature=emb_logo}
    \item \url{https://www.youtube.com/watch?v=uK4j8CtIXgs&feature=emb_logo}
\end{enumerate}

% Some text about tutorial

% \section*{Robust Deep Learning Methods for Anomaly Detection} 

% Some text about tutorial

