\chapter{SimClusters: Community-Based Representations for Heterogeneous Recs at Twitter}

Twitter \\

\textbf{Refernce:}~\url{https://dl.acm.org/doi/pdf/10.1145/3394486.3403370}

\textbf{Keywords:} user behavior, nearest neighbours, clustering

\section*{Какую задачу решают авторы?}

\section*{Как решают?}

\subsection*{Stage 1: Community Discovery}

\paragraph{Similarity Graph of Right Nodes}

\paragraph{Communities of Right Nodes}

\paragraph{Communities of Left Nodes}

\subsection*{Stage 2: Item Representations}

\subsection*{Applications}

Предложенная система работает в Twitter уже около года и за это время нашла себе применение в разных продуктах компании.

В рамках каждого из продуктов, новая система показала ощутимый рост метрик в сравнении с существующими методами.

\section*{Преимущества подхода}

\dbox{\textbf{Key Takeaway}:
In our experience of building recommendations at Twitter, we find that neighborhood-based methods are easier to scale, more accurate, more interpretable, and also more flexible in terms of accommodating new users and/or items
}