\begin{bibunit}[plainnat]

\chapter{On Sampled Metrics for Item Recommendation}

Google Research \\

\textbf{Best research paper award}

\textbf{Reference:}~\url{http://walid.krichene.net/papers/KDD-sampled-metrics.pdf}

\textbf{Keywords:} ranking evaluation, metrics, sampling

\section*{Какую задачу решают авторы?}

Типичный протокол оценки качества рекомендательных систем выглядит следующим образом:
\begin{enumerate}
    \item Для отобранного множества пользователей $D$ ранжируем алгоритмом $A$ все множество кандидатов, состоящее из $n$ объектов
    \item Для каждого пользователя $\vecx$ вычисляем $R(A, \vecx)$ --- множество позиций релевантных объектов
    \item После чего для пользователя считаем метрику $M$, например, \texttt{ROC AUC} или \texttt{Precision@K}, \texttt{Recall@K}
\end{enumerate}

Итоговое значение метрики получается усредненеим метрик посчитанных по всем пользователям
\begin{equation*}
    \frac{1}{|D|}\sum\limits_{\vecx \in D} M(R(A, \vecx)) .
\end{equation*}

В ситуации когда $n$ велико часто прибегают к сэмплированию --- вместо того чтобы ранжировать все $n$ кандидатов, ранжируют случайную подвыборку из $m$ объектов ($ m \ll n $) вместе с релевантными для пользователя объектами. \\

Ожидается, что метрики посчитанные с сэмплированием позволяют упорядочить алгоритмы ранжирования по качеству так же как и метрики посчитанные без сэмплирования. 

Авторы статьи впервые тестируют это предположение и показывают что для большинства используемых метрик оно не верно, даже при многократном сэмплировании и усреднении результатов. \\

В статье предложены скорректированные варианты привычных метрик, которые позволяют при использовании сэмплирования сортировать алгоритмы по качеству также как если бы сэмплирования не было. 

\section*{Как решают?}

\textbf{Remark:} для простоты можно считать, что для каждого пользователя есть один релевантный объект. \\

Работу можно разбить на три части
\begin{enumerate}
    \item Экспериментальная часть, где проверяют предположение, что метрики с сэмплированием и без упорядочивают алгоритмы ранжирования одинаковым образом
    \item Теоретическая часть, посвященная скорректированным вариантам метрик
    \item Экспериментальная часть, где авторы показывают, что скорректированные варианты метрик работают
\end{enumerate}

\subsection*{Inconsistency of Sampled Metrics}

Ключевое определение данной работы

\ddef{Consistency}{
\label{def:consistenct}
Let the evaluation data $D$ be fixed. A metric $M$ is consistent under sampling if the relative order of any two recommenders $A$ and $B$ is preserved in expectation. That is, for all $A,B$,
\begin{align*}
    & \frac{1}{|D|} \sum\limits_{\vecx \in D} M(R(A,\vecx)) > \frac{1}{|D|} \sum\limits_{\vecx \in D} M(R(B,\vecx)) \\
    \Longleftrightarrow & \bE \left[ \frac{1}{|D|} \sum\limits_{\vecx \in D} M(\widetilde{R}(A,\vecx))  \right] > \bE \left[ \frac{1}{|D|} \sum\limits_{\vecx \in D} M(\widetilde{R}(B,\vecx))  \right]
\end{align*}
}

В рамках первой группы экспериментов (см. Рисунок~\ref{fig:example_vary_m}) показывают как меняются метрики для разных алгоритмов в зависимости от $m$ (число случайно отобранных объектов). \\

\begin{figure}[ht]
    \centering
    \includegraphics[width=0.49\textwidth]{figures/ex_num_samples_vs_ap.pdf}
    \includegraphics[width=0.49\textwidth]{figures/ex_num_samples_vs_ndcg.pdf} \newline
    \includegraphics[width=0.49\textwidth]{figures/ex_num_samples_vs_recall.pdf}
    \includegraphics[width=0.49\textwidth]{figures/ex_num_samples_vs_auc.pdf}
    \caption{\footnotesize{Expected sampling metrics for the running example while increasing the number of samples.
    For Average Precision, NDCG and Recall, even the relative order of algorithm performance changes with the number of samples.
    That means, conclusions drawn from a subsample are not consistent with the true performance of the algorithm.}}
    \label{fig:example_vary_m}
\end{figure}

Из экспериментов видно, что единственная консистентная метрика --- \texttt{ROC AUC}.

\subsection*{Corrected Metrics}

Очень рекомендую самостоятельно ознакомиться с данной частью работы, так как ее не очень удобно излогать в виде конспекта.

\section*{Выводы}

\dbox{\textbf{Key Takeaways}:
\begin{enumerate}
    \item Нужно избегать сэмплирования при расчете метрик
    \item Если нет возможности избежать сэмплирования, то нужно использовать скорректированные версии метрик
    \item Вычисление метрик нужно повторять несколько раз с разным seed'ом для того чтобы уменьшить дисперсию
\end{enumerate}}

Все это в лишний раз наводит на мысли о том, что в ряде работ лучшими оказались алгоритмы, которые не обязательно являются лучшими на самом деле, и на результаты экспериментов в статьях всегда надо смотреть с определенной долей скепсиса.

\chapter{Temporal-Contextual Recommendation in Real-Time}

Amazon \\

\textbf{Best ADS paper}

\textbf{Reference:}~\url{https://dl.acm.org/doi/pdf/10.1145/3394486.3403278}

\textbf{Keywords:} recommender systems, recurrent neural networks, hybrid model

\section{Какую задачу решают авторы?}

Разработчики современных рекомендательных систем сталкиваются со следующими челенджами: Система должна

\begin{itemize}
    \item Оперативно реагировать на изменение интересов пользователя
    \item Обучаться на историях большого числа пользователей, состоящих из сотен событий, за разумное время
    \item Быть эффективной для новых пользователей и объектов (cold-start problem)
    \item Хорошо масштабироваться на случай ранжирования большого числа объектов
\end{itemize}

Многие популярные решения не удовлетворяют всем указанным требованиям. 

Например, решения, в основе которых лежит факторизация матрицы \texttt{Users-Items}, не позволяют оперативно реагировать на изменение интересов пользователя, и не могут строить рекомендации для новых пользователей/объектов. \\

В статье авторы предлагают решение, которое удовлетворяет всем перечисленным требованиям.

В основе решения --- \texttt{RNN-like} архитектура, которая наряду с идентификаторами пользователя и объекта использует доступную мета-информацию (признаки пользователя и объекта).

Для того чтобы модель можно было за разумное время обучить на большом каталоге объектов, авторы предлагают использовать \texttt{Negative Sampling}~\cite{mikolov2013distributed} (NS) вместо классической многоклассовой классификации.

\section{Как решают?}

Не смотря на то, что в статье описано много интересного, я хотел бы уделить внимание только одному моменту в работе --- использование NS для ускорения обучения.

\subsection{Negative Sampling}

Сначала ответим на вопрос почему обучение моделей рекомендаций при большом размере каталога товаров может быть медленным. \\

Рассмотрим следующую постановку задачи рекомендаций: пусть у нас есть история пользователя $ \{i_1, \ldots, i_k \} $, например, какие товары он уже купил, наша задача научиться предсказывать следующий товар $i_{k+1}$, который заинтерисует пользователя. \\

Довольно часто задачу в такой постановке решают сводя ее к задаче \textit{многоклассовой классификации}, где классам соответствуют товары из каталога. 

Детальнее, пусть каталог состоит из $m$ товаров,
\begin{enumerate}
    \item Используя модель, мы получаем скрытое представление $\vech_k \coloneqq \vech(i_1, \ldots, i_k) \in \bR^{d} $, описывающее историю пользователя
    \item Полученное представление $\vech_k$ домножаем на матрицу $W \in \bR^{m \times d}$, чтобы получить скор для каждого товара в каталоге
    \item Итоговые вероятности для товаров 
        \begin{equation*}
            \hat{y} \coloneqq \softmax (W \cdot \vech_k)
        \end{equation*}
\end{enumerate}

Модель $\vech$ и матрица $W$ обучаются оптимизируя \texttt{Cross Entropy Loss}. \\

Домножение вектора $\vech_k$ на матрицу $W$ с последующим softmax'ом занимает много времени, кроме того, матрица $W$ занимает очень много места. \\

При использовании NS~\cite{mikolov2013distributed} нам не нужно вычислять скоры для всех товаров, а только для релевантного товара и для небольшого случайного множества негативных примеров, так как решаем задачу \textit{бинарной классификации} --- учимся отличать релевантные товары от нерелавнтных.

Это позволяет не хранить матрицу $W$ и существенно ускоряет процесс обучения.

\subsection{Experiments}

В рамках экспериментов авторы исследуют то как зависит скорость обучения (\#items/sec) при использовании NS от размера каталога. \\

Авторы замерели скорость обучения и метрики на датасетах MovieLens (1m, 10m, 20m) и на датасете Taobao, при использовании Dense output слоя и при использовании NS.

На датасетах MovieLens модель обучали в течении четырех эпох. На датасете Taobao время на обучение было ограничено шестью часами. \\

Исходя из результатов (см. таблицу~\ref{table:temporal}), можно сделать вывод, что использование NS существенно ускоряет обучение в случае большого каталога объектов (Taobao).

Кроме того, при ограничениях на время обучения, использование NS позволяет добиться лучшего качества за счет того, что модель успевает обработать большее число объектов. 

\begin{figure}
    \centering
    \begin{tabular}{cccccc}
        \hline & Output & ml-1m & ml-10m & ml-20m & Taobao \\
        \hline Output size & IS & 62 & 255 & 362 & 1087 \\
        $(m)$ & Dense & 1683 & 65134 & 131263 & 1183451 \\
        \hline Throughput & NS & $\underline{23 \mathrm{k}}$ & $20 \mathrm{k}$ & $\mathbf{2 0 k}$ & $\mathbf{7 . 8 k}$ \\
        (\#items/sec) & Dense & $23 \mathrm{k}$ & $23 \mathrm{k}$ & $17 \mathrm{k}$ & 631 \\
        \hline PPL & NS & $\underline{377}$ & $\underline{405}$ & $\underline{455}$ & $\mathbf{1 7 . 6 k}$ \\
        & Dense & 409 & 439 & 494 & $119 \mathrm{k}$ \\
        \hline NDCG & NS & 0.128 & 0.123 & 0.12 & $\mathbf{0 . 1 5}$ \\
        & Dense & 0.123 & 0.119 & 0.115 & 0.08 \\
        \hline
    \end{tabular}
    \caption{\footnotesize{Model throughput and accuracy. Bold-fonts/underlines show significant/insignificant differences, respectively. NS significantly improved model training for up to 1M unique items (Taobao dataset). Additionally, the dense model for the Taobao dataset requires >16GB memory for BPTT, which is infeasible on many GPUs.}}
    \label{table:temporal}
\end{figure}

\newpage

Наряду с привычными метриками для данной задачи, в статье используют еще и \textbf{Perplexity}.

\textbf{Remark:} A recommendation system with a PPL of $p$ is equivalent to one that recommends a uniform random selection of $p$ items, one of which is the true next item.

\section{Выводы}

\dbox{\textbf{Key Takeaways}:
\begin{enumerate}
    \item Если каталог объектов очень большой ($\geq 10^5$), то для обучения модели ранжировани лучше использовать \texttt{Negative Sampling}.
    Это позволяет не только ускорить обучение, но и существенно уменьшить размер модели.
\end{enumerate}}

% \chapter{Joint Optimization of Multiple Objectives on Music Streaming Platforms}
% \cite{rishabh2020multiobjective}

% \chapter{SimClusters: Community-Based Representations for Heterogeneous Recs at Twitter}
% \url{https://dl.acm.org/doi/pdf/10.1145/3394486.3403370}

% \chapter{Managing Diversity in Airbnb Search}
% \cite{abdool2020managing}

% \chapter{Multitask Mixture of Sequential Experts for User Activity Streams}

\chapter{Learning to Cluster Documents into Workspaces Using Large Scale Activity Logs}

Google \\

\textbf{Refernce:}~\url{https://dl.acm.org/doi/abs/10.1145/3394486.3403291}~\cite{kong2020learning}

\textbf{Keywords:} user behavior, embeddings, clustering

\section*{Какую задачу решают авторы?}

Авторы решают задачу кластеризации документов в Google Drive в workspace'ы --- отдельные папки, содержащие документы похожие не только по смыслу, но и связанные с конкретными задачами пользователя. \\

У классических подходов к кластеризации документов есть недостатки связанные с тем, что они могут группировать вместе документы одной тематики, но при этом они не учитывают большое количество других факторов.

Например, если пользователь работает над конкретной задачей, то он вполне может использовать документы с разной тематикой. \\

В рамках данной работы, авторы не решают задачу кластеризации напрямую, но обучают document similarity модель, которая для двух документов предсказывает относятся они к одному кластеру или нет.

Используя предсказания модели, авторы используют иерархическую кластеризацию для группировки документов в workspace'ы. \\

\textit{Главный вопрос:} где взять данные для обучение document similarity модели?

На практике, получение датасета достаточного размера, состоящего из пар документов с разметкой от пользователей, может быть довольно сложно. \\

Авторы предлагают способ получить разметку для обучения напрямую из логов активности пользователей --- сводят задачу к weakly supervised варианту.

Основное предположение, которое позволяет получить weak разметку --- если пользователь выполнял действия с документами с небольшой разницей во времени (\textit{co-access}), то скорее всего эти документы относятся к одной задаче, над которой работает пользователь. \\

Авторы в онлайне сравнивают свое решение с классическими unsupervised подходами для кластеризации документов, и показывают, что предлагаемое решение существенно лучше бэйслайнов.

\section*{Как решают?}

Рассмотрим сначала детальнее то как авторы предлагают получать weak разметку, а затем на предлагаемую в статье document similarity модель.

\subsection*{Activity-Based Labels and Weak Supervision}

Для получения разметки, авторы отталкиваются от предположения о похожести документов, с которыми пользователь работал в течении небольшого промежутка времени.

Два документа считаются co-accessed, и получают $label = 1$, если были открыты с разницей во времени не больше двух минут. \\

\begin{wrapfigure}{r}{0.5\textwidth}
    \includegraphics[width=0.99\linewidth]{images/dsm_labels.png}
    \caption{\footnotesize{Past and future activity segments used for extracting activity-based features and labels respectively}}
    \label{fig:dsm_labels}
\end{wrapfigure}
На рисунке~\ref{fig:dsm_labels} показано на каких данных обучается document similarity модель. \\

По сути, вся активность пользователя делиться на две части: первая используется для построения признаков и обучения document similarity модели, а вторая --- для получения разметки. \\

Документы считаются похожими, если они были co-accessed в рамках второй части активности пользователя.

\subsection*{Document Similarity Model}

Рассмотрим подробнее как выглядит модель (см. Рисунок~\ref{fig:dsm}).

\begin{figure}[ht]
    \centering
    \includegraphics[width=0.7\linewidth]{images/dsm.png}
    \caption{\footnotesize{Document Similarity Model}}
    \label{fig:dsm}
\end{figure}

Список признаков, на которых обучается модель:

\begin{itemize}
    \item \textbf{Text features}
        \begin{itemize}
            \item $text(d)$ - text content for document $d$
            \item $text(d')$ - text content for document $d'$
        \end{itemize}
    \item \textbf{Metadata features}
        \begin{itemize}
            \item $mime(d)$ - MIME type of the document $d$
            \item $mime(d')$ - MIME type of the document $d'$
        \end{itemize}
    \item \textbf{Activity-based features}
        \begin{itemize}
            \item $recent\_co\_accesses(d, d')$ - number of co-accesses between $d, d'$ in the past 2 weeks
            \item $historic\_co\_accesses(d, d')$ - number of co-accesses between $d, d'$ in the past 4 weeks
        \end{itemize}
    \item \textbf{Human and activity-based labels}
        \begin{itemize}
            \item $co\_cluster(d, d')$ - human labels on whether $d, d'$ should be clustered together in a workspace
            \item $future\_co\_accesses(d, d')$ - number of co-accesses between $d, d'$ in the future week
        \end{itemize}
\end{itemize}

Авторы показывают, что дополнительное использование Activity-based  признаков существенно улучшает качество модели, то есть полезно использовать не только признаки связанные с документами, но и информацию о том как пользователь с ними взаимодействовал. \\

DSM обучается минимизируя, привычный для задачи бинарной классификации, Binary Cross-Entropy Loss.

\subsection*{Experiments}

Авторы удтверждают, что documents similarity модель обученная на weak разметке, по качеству не уступает моделе, обученной на разметке от пользователей. \\

Кроме того, авторы проводят онлайн эксперимент и оценивают качество предложенного алгоритма кластеризации со следующими бэйслайнами:

\begin{itemize}
    \item Topicality - для документов обучаются эмбеддинги, которые потом кластеризуются
    \item Calendar - относит в один кластер документы, прикрепленные к одному событию в календаре
    \item Favorites - группирует вместе документы, которые пользователь открывает наиболее часто
\end{itemize}

По результатам экспериментов, предложенный подход работает намного лучше бэйслайнов.

\section*{Мое мнение}

Красивый и простой способ извлечения weak лэйблов, для обучения модели похожести документов, из большого количества логов активности пользователя.

\chapter{Другие работы}

\section*{Embedding-based Retrieval in Facebook Search}

Facebook \\

\textbf{Reference:}~\url{https://arxiv.org/abs/2006.11632}

\textbf{Конспект:}~\url{https://vk.com/@papersreaders-embedding-based-retrieval-in-facebook-search} \\

Исторически поиск в Facebook работал на основе Boolean matching model.

В статье авторы делятся опытом перехода к использованию embedding-based системы на этапе отбора кандидатов перед ранжированием. \\

В основе предлагаемого решения модель построения эмбеддингов запросов и документов. \\

Работа наглядно показывает, что даже относительно несложное решение для Embedding-based retrieval показывает существенное улучшение в сравнении с классическим Boolean matching подходом в задаче отбора кандидатов.

\section*{Controllable Multi-Interest Framework for Recommendation}

Alibaba \\

\textbf{Reference:}~\url{https://arxiv.org/abs/2005.09347}

\textbf{Конспект:}~\url{https://vk.com/@papersreaders-controllable-multi-interest-framework-for-recommendation} \\

Современные рекомендательные системы используют историю пользователя для построения вектора, который описывает пользователя и используется для поиска объектов-кандидатов при построении рекомендаций. 

Однако использование единственного вектора для описания пользователя не позволяет уловить его разнообразные интересы. \\

Авторы статьи предлагают решение, которое позволяет представить пользователя в виде набора из К векторов, каждый из которых соответствует некоторому интересу пользователя.

Данный подход позволяет делать рекомендации как более точными так и более разнообразными в сравнении с существующими state-of-the-art подходами.

\section*{PinnerSage: Multi-Modal User Embedding Framework \\ for Recommendations at Pinterest}

Pinterest \\

\textbf{Reference:}~\url{https://arxiv.org/pdf/2007.03634.pdf} \\

Статья от Pinterest про прокачку системы рекомендаций пинов для пользователя. \\

Авторы статьи рассматривают проблемы связанные с представлением пользователя в виде единственного вектора.

Для решения проблем, в статье предлагают представить пользователя в виде набора векторов. \\

Ключевые отличия от предыдущих работ, предлагающих сделать тоже самое:
\begin{enumerate}
    \item количество векторов для пользователя не фиксировано
    \item вектора для пользователей не обучаются совместно с векторами для пинов
\end{enumerate}

Для того чтобы представить пользователя в виде набора векторов, предлагают делать иерархическую кластеризацию активности пользователя за последнее время (вектора пинов получены black-box моделью).

Каждому кластеру ставят в соответствие его важность.  \\

Для рекомендации релевантных пинов берут 3 наиболее важных кластера и ищут похожие пины с помощью приближенного поиска ближ соседей. \\

Как и в большинстве последних статей от Pinterest, авторы рассматривают продакшн решение, поэтому достаточно внимания уделяют вопросу о том как все это тащить в прод.

\section*{Improving Recommendation Quality in Google Drive}

Google \\

\textbf{Reference:}~\url{https://research.google/pubs/pub49272/} \\

Статья про попытки прокачать качество инструмента Quick Access (статья не научная, а просто про опыт и проведенные эксперименты) \\

В целом, то о чем они пишут очень похоже на наш опыт и на наши эксперименты за последние года полтора. \\

Эксперименты описанные в статье, которые похожие на наши
\begin{itemize}
    \item автоматизация пайплайнов переобучения моделей
    \item попытки затащить DL модели (в статье наоборот пытаются перейти к GBDT)
    \item изучение того как влияет latency на метрики
    \item несколько примеров feature engineering'a 
    \item эксперимент с увеличением числа кандидатов для ранжирования
    \item фича мониторинг
\end{itemize}

\addcontentsline{toc}{chapter}{Литература}
\putbib[refs_recsys]
\end{bibunit}
